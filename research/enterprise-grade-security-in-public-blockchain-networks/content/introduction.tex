% INTRODUCTION
\section{Introduction}
The focus of this research is to analyze how security is managed in public blockchain networks, with an emphasis on techniques that enable high-level security and privacy. The findings provides implementation possibilities, demonstrating practical methods and strategies for integrating security and privacy techniques into existing and new blockchain infrastructures. Each blockchain network typically employs a unique security model, generally comprising various critical components, including \textit{cryptographic mechanisms}, \textit{consensus algorithms}, \textit{network security protocols}, \textit{identity and access management}, \textit{verification and auditing processes}, \textit{privacy measures}, \textit{isolated execution environments}, \textit{monitoring and response systems}, and \textit{regulatory compliance strategies}.

In this research paper, we conduct a detailed examination of previously mentioned aspects to develop a solution that enables private and hybrid blockchain networks to be operated securely and entirely trustless by the public participants. The objective is to formulate robust strategies and implementation methodologies that ensure the security and integrity of public blockchain networks. This includes maintaining transparency and trustlessness, even when these networks are publicly accessible, thereby enhancing their overall reliability and adoption.

The research itself focuses on general aspects of security in public blockchain networks. However, the proposed solution (findings section) and the subsequent implementation of recommended practices will be integrated into the Substrate framework. This will enable a security model that can be specifically implemented for Substrate-based chains. While the security model can, in general terms, be applicable across various frameworks, our primary focus will be on its implementation within Substrate.

A security system that does not implement principles of \textit{zero-trust}, \textit{zero-knowledge}, and \textit{zero-tolerance} cannot provide complete security. Each of these principles plays a crucial role in ensuring robust security measures: zero-trust assumes no implicit trust in any network component, zero-knowledge ensures privacy through cryptographic proofs without revealing data, and zero-tolerance mandates strict enforcement of security policies. Our approach is to "fully" integrate these principles, thereby achieving a comprehensive and robust security model.

An enterprise-grade security model aims to ensure the highest security standards to effectively counteract threats. Key differentiators of an enterprise-grade solution include scalability to handle large volumes of data and transactions, advanced threat detection and mitigation, compliance with legal and regulatory requirements, ensuring data integrity and availability. Implementing an enterprise-grade security solution makes public governed blockchains more robust, trustworthy, and resilient against threats, which is crucial for its long-term adoption and use.

% MOTIVATION
\subsection{Motivation}
Security in blockchain networks is one of the most crucial factors for the broader adoption of blockchain solutions. Most public blockchain platforms are rarely or never used for enterprise solutions because private blockchain platforms (such as IBM Blockchain Platform, R3 Corda Enterprise, Kaleido, Azure Blockchain Workbench, etc.) are generally perceived as more secure and are utilized for many applications. One of the key differentiating factors is the enterprise-grade security and privacy. While public blockchain platforms offer a certain level of security and privacy, they are often not suitable for many use cases that require a higher degree of protection. Enterprise blockchain platforms typically come with higher costs, making migration unaffordable for many Web2 services. Public blockchain platforms are more cost-effective but are often viewed as not equally secure.

The possibility of providing a fully community-operated blockchain infrastructure, where completely private services with high-security requirements can be run, would present a viable path for migrating from Web2 to Web3. This approach can enable for example services from various sectors such as public transportation, communal services, educational institutions, and cultural institutions to be operated and governed by the community, advancing the migration to Web3. The implementation of private elements in a public blockchain network aims to balance transparency and data privacy, ensuring that data can be handled securely and privately even in a publicly provisioned infrastructure.

The primary reason for this research is the provision of the \textit{Logos network} \cite{AboutLogosNetworkDocs} and its resulting \textit{sub0layer} \cite{sub0layerBlog}. The approach is to provide a community enterprise grade computation and storage solution (a DePin - Decentralized Physical Infrastructure Network) for the Web3 based on the blockchain technology, which means that a blockchain network must contain private elements but can still be operated securely by the community. The security and privacy requirements that such a network entails will be further discussed in Chapter 2, Case Study, where the specific requirements will be addressed in detail.

Our primary driving factor is to enable the provision of resources and infrastructure for Web3 through a community-driven, secure, and trustless model by implementing new security principles. This approach adheres to the true ethos of Web3, ensuring that no private parties (e.g., private validators) centralize the network or undermine its trustless nature.   

% DOCUMENT STRUCTURE
\subsection{Document Structure}
The research paper is structured to be released in multiple stages. The rationale behind this approach is to divide the security model into different sections, allowing various methods to be addressed separately. This enables individual researchers and developers to focus on a specific topic. Once the research paper is fully developed, a comprehensive release, version 1.0, will be published.

By addressing individual topics separately, the findings and forthcoming implementations can be provided independently. In the Substrate ecosystem, functionalities are provided through separate modules, known as pallets. This means that developments in areas such as Identity and Access Management can take place first, while other areas like consensus algorithms or isolated execution environments can be addressed subsequently or concurrently.

It is important to emphasize that this separation is feasible because the main requirements or rules are clearly defined, and each implementation area must adhere to these primary requirements.

The paper is divided into six main chapters. The first chapter (Introduction) provides an overview of the topics covered and the motivations driving this research. In the second chapter (Case Study), we define the primary requirements of a secure public blockchain network that demands full zero-trust principles. In this section, we divide the security model into separate sections according to the specific areas they address.

Chapters 3 (Theoretical Foundations), 4 (Findings), and 5 (Discussion) are subdivided according to the sections defined in Chapter 2. The third chapter (Theoretical Foundations) examines current solutions and implementations for each defined section of the security model. In Chapter 4 (Findings), we present the results and potential implementations. Chapter 5 (Discussion) explains the significance of the achieved results and how they should be interpreted in relation to existing solutions.

The final chapter (Conclusion) summarizes the key findings of the study and emphasizes the main contributions of the research to the field. Additionally, it discusses potential practical applications of the research findings. The paper concludes with a bibliography and a glossary of key terms related to the research.